\documentclass[10pt, a4paper]{article}
\usepackage[T1]{fontenc}
\usepackage[utf8]{inputenc}
\usepackage[a4paper, left=1cm, right=1.6cm, top=1cm, bottom=1.6cm]{geometry}
\usepackage[english]{babel}
\usepackage{blindtext}
\usepackage[slc=on]{caption} %Align caption to table
\usepackage{threeparttable, booktabs} % booktabs for \bottomrule\toprule\midrule
\usepackage{multirow}
\usepackage[table]{xcolor}
\usepackage{lipsum}
\usepackage{array}
\usepackage{lscape}

\setlength{\abovecaptionskip}{3pt plus 3pt minus 2pt} 


%\setlength{\tabcolsep}{8pt}
\renewcommand{\arraystretch}{1.1} %Spacing between rows

\setlength{\parindent}{0pt}

\begin{document}
\listoftables

\section{Ein erstes Beispiel}
Table in the middle of running text.
\begin{tabular}[c]{c|c}
\hline
1 & 2 \\
\hline 
3 & 4\\
\hline
\end{tabular}. This table is in the running text.


\section{Multirow und -column}

\setlength{\tabcolsep}{5pt} %Spacing between columns

\subsection{Multicolumn}

% >{} uses the array package. Inserts the content of {} at the beginning of each cell in this column. Conversely <{} inserts the content in {} at the end of each cell in this column.
% E.g.. >{$}c{$}< - centered column which starts with $ and ends with $.
\begin{tabular}{lc>{\centering}p{5cm}c}
\hline
\textbf{\LaTeX-command} & \multicolumn{3}{c}{\textbf{Base font size}}\\
\cline{2-4} & \textbf{10pt} & \textbf{11pt} & \textbf{12pt}\\
\verb_\tiny_ & 5pt & 6pt & 6pt\\
\hline
\end{tabular}

\begin{tabular}{>{$}c<{$}}
\mathcal{C}
\end{tabular}

\subsection{Multirow}

\begin{tabular}{lll}
\hline
\multirow{1}{*}{Multirow} & 1 & 2 \\
 & a & b \\
 & c & d \\
\hline
\end{tabular}

\subsection{Multirow and -column combined}

\begin{tabular}{lccc}
\hline
\multirow{2}{5cm}{\centering\textbf{\LaTeX-command}} & \multicolumn{3}{c}{\textbf{Base font size}} \\
& \textbf{10pt} & \textbf{11pt} & \textbf{12pt} \\
\hline
\end{tabular}

\subsection{Cline}

\begin{tabular}{|c|c|c|c|}
\hline
1 & 2 & 3 & 4\\
\cline{2-4} d & 10 & 11 & 12\\
\hline
\end{tabular}

{\centering
\rowcolors{2}{white}{red!20}
\begin{tabular}{ l | p{10cm}  r }
  \textbf{a} & \textbf{b} & \textbf{c} \\
  \hline \hline
  16:00 -- 17:00 & \lipsum[1] & blubb \\
  17:00 -- 18:00 & More text & The end \\
  18:00 -- 19:00 & \multicolumn{2}{r}{Finally dinner} \\
\end{tabular}\\[1em]
}
\section{Float environments}

\begin{table}[h]
\caption{Test}\label{tab:baggins}
\begin{tabular}{ll}
\hline
Name: & Baggins \\
First Name: & Bilbo\\
Address: & Bagshot Row 56, 10982 Hobbiton\\
\hline
\end{tabular}
\end{table}


\begin{table}[htbp]
\centering\renewcommand\arraystretch{1.1}
\begin{threeparttable}
\caption{\blindtext}
\begin{tabular}{ll}
\toprule
Name: & Baggins \\
First Name: & Bilbo \\
Address: & Bagshot Row 56, 10982 Hobbiton \\
\bottomrule
\end{tabular}
\end{threeparttable}
\end{table}

\begin{landscape}
  {
\rowcolors{2}{white}{red!20}
\begin{tabular}{ l  >{\centering}p{20cm}  r }
  \textbf{a} & \textbf{b} & \textbf{c} \\
  \hline \hline
  16:00 -- 17:00 & \lipsum[1] & blubb \\
  17:00 -- 18:00 & Mehr Text & Endlich Ende \\
  18:00 -- 19:00 & \multicolumn{2}{r}{Endlich Abendessenszeit} \\
\end{tabular}\\[1em]
}
\end{landscape}

\begin{table}[ht]
  \centering
  \caption{Revisions}
   \begin{threeparttable}
  \centering
      \begin{tabular}{cccc}
      \toprule
          Title 1 & Title 2 & Title 3 & Title 4          \\
      \midrule
          Cell 1  & Cell 1  & Cell 3  & Cell 4 \tnote{a} \\
          Cell 1  & Cell 1  & Cell 3  & Cell 4 \tnote{b} \\
      \bottomrule
      \end{tabular}
      \begin{tablenotes}
          \item[a] My Note.
          \item[b] My Other Note.
      \end{tablenotes}
   \end{threeparttable}
\end{table}

\end{document}
