\documentclass[10pt, a4paper]{article}
\usepackage[T1]{fontenc} 
\usepackage[utf8]{inputenc}
\usepackage[ngerman]{babel}

\setlength{\parindent}{0pt}
\begin{document}
``Test''  - englische Form\\
"`Test"'  - deutsche Form oder \glqq So\grqq \\ %alternativ \glq \grq (braucht [ngerman]{babel}
"<Test"> - franz"osische Form oder \flqq So\frqq %alternativ \flq \frq

\vspace{1cm}
Silbentrennung:

\verb+\-+ \quad Wort wird nur an der Stelle getrennt werden k"onnen, wo dies so gekennzeichnet ist. Andere
Trennstellen werden vom Trennalgorithmus ignoriert.

\verb+"-+ \quad Stellt eine zus"atzliche Trennstelle her. Bei zusammengesetzten Nomen ist das hilfreich.
Bsp: \verb+Schulhaus"-verwaltung+

\verb+""+ \quad Definiert einen m"oglichen Zeilenwechsel wo dann kein zus"atzlicher Trennstrich eingef"ugt wird.
Bsp: Di-""Methyl-""Aceton.
\end{document}
