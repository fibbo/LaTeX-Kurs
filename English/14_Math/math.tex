\documentclass[10pt, a4paper]{article}
\usepackage[T1]{fontenc}
\usepackage[utf8]{inputenc}
\usepackage{url}

% Mathpackages
\usepackage{mathtools} 
\usepackage{amsfonts}
\usepackage{amssymb}

\usepackage{hyperref}
\numberwithin{equation}{section}
\numberwithin{figure}{section}

\setlength{\parindent}{0pt}
\begin{document}

\section{Math Stuff}

\url{http://detexify.kirelabs.org/classify.html}

Simple mathematical equation: \( 2 + 3x = 4\) Placed within running text.

Explicitly emphasize a formula.

\[ 2x + 3x = 4 \]

Again emphasized but now also numbered:
\begin{equation}
2x + 3x = 4
\end{equation}

\subsection{Powers and indices}

\[ 2^x + 3a_i + c^{x+y}d_{ij} = z  \]


Leading powers and indices:
\[ {}^3_2\mathrm{A} \]
\[ {}^{18}_{\phantom{1}9}\mathrm{F} \]


\subsection{Fractions}

\[\frac{1}{2}, \quad \frac{x}{yz},\quad \frac{123}{456}\]

\subsubsection{Multifractions}

Multifractions get squashed when placed inside text. \( \frac{\frac{a}{x-y}+\frac{b}{x+y}}{1+\frac{a-b}{a+b}} = 1. \) Using \texttt{dfrac} can increase spacing within the fractions: \( \qquad \dfrac{\frac{a}{x-y}+\frac{b}{x+y}}{1+\frac{a-b}{a+b}} = 1 \). It's better to place mathematical expressions apart from the rest of the text.

\[\frac{\frac{a}{x-y}+\frac{b}{x+y}}{1+\frac{a-b}{a+b}} = 1 \qquad \dfrac{\frac{a}{x-y}+\frac{b}{x+y}}{1+\frac{a-b}{a+b}} = 1 \]

\subsection{Roots}

\[ \sqrt{x},\quad \sqrt[n]{x}  \]

\subsection{Matrices}

\begin{equation}
\begin{bmatrix}
  a & b & c \\
  d & e & f \\
  g & h & i
 \end{bmatrix}
\end{equation}

\subsection{Text in the math environment}
\( x^2 \geq 0\quad für alle x \leq 0 \)\\ %Treat text like a variable
\( x^2 \geq 0\quad \mathrm{für alle } x \leq 0 \)\\ %Treat text as variable but not cursive (roman)
\( x^2 \geq 0\quad \mbox{für alle } x \leq 0 \)\\ %respects spaces and umlaute
\( x^2 \geq 0\quad \text{für alle } x \leq 0 \)\\ %best suited for powers and indices 
Superscripted text with mbox: \quad \( y^{\mbox{Power}}=27\)\\
Subscripted text with text: \quad \( y^{\text{Power}}=27\)

\subsection{General mathematical functions}

\( \int_{0}^{\frac{\pi}{2}} \sin x\,\mathrm{d}x \qquad
\int\limits_{0}^{\frac{\pi}{2}} \sin x\,\mathrm{d}x \qquad\qquad
\int_{-\infty}^{+\infty}x^2\,\mathrm{d}x \qquad
\int\limits_{-\infty}^{+\infty}x^2\,\mathrm{d}x\)

\[ \int_{0}^{\frac{\pi}{2}} \sin x\,\mathrm{d}x \qquad
\int\limits_{0}^{\frac{\pi}{2}} \sin x\,\mathrm{d}x \qquad\qquad
\int_{-\infty}^{+\infty}x^2\,\mathrm{d}x \qquad
\int\limits_{-\infty}^{+\infty}x^2\,\mathrm{d}x\]


%Do not use eqnarray!
%\begin{eqnarray}
%a & = & 2x \\
%0 & = & 3
%\end{eqnarray}

\begin{equation}
 	a^2 + b^2 = c^2\label{eq:a}
\end{equation}


%Suprress numbering of equation with \nonumber 
\begin{align}
  (a+b)^2    &=    a^2+2ab+b^2 \nonumber \\
  (a-b)^2    &=    a^2-2ab+b^2 \nonumber \\
  (a+b)(a-b) &=    a^2-b^2
\end{align}

\subsection{A more complicated expression}
\[ \rho\left( \frac{\partial \mathbf{v}}{\partial t} + \mathbf{v}\cdot\nabla\mathbf{v}\right) = -\nabla p + \nabla\cdot\mathsf{T}+\mathbf{f} \]

\[ \overbrace{x^2 + \underbrace{y^2}_\text{part of the equation}}^\text{whole equation} \]

\[ \overbrace{x^2 + \underbrace{y^2}_{\mathclap{\text{part of the equation}}}}^{{\text{whole equation}}} \]

\begin{figure}[h!]
Given a family of functions

\[ f_a(x) = \frac{x+a}{x^2} \text{mit } a \in \mathbb{R} \]

\begin{itemize}
\item Investigate the family of functions \( f_a \) for their domain \( \mathbb{D} \).
\item \( x \to \infty \)
\item Show the graph \( G_{f_1} \) in the intervall \( I = [-4,4] \)
\item Show that 
\[ F(x) = x+ (x+1)\cdot\ln(x+1)-2x\cdot\ln(x) \]
the antiderivative of \[ g(x) = \ln\left(\dfrac{x+1}{x^2}\right) \] is
\item Determine one \ldots

\[ F_x(x) = \int\limits_2^x\ln\left(f_1(t)\right)\mathrm{d}t \]

\end{itemize}
\end{figure}

\url{http://en.wikibooks.org/wiki/LaTeX/Mathematics}
\end{document}
