\documentclass[10pt, a4paper]{beamer} %handout fuer eine gedruckte version der slides
\usetheme{metropolis}

\usepackage{url}
\usepackage{polyglossia}
\setmainlanguage{english}

\usepackage{listings,xcolor}
\usepackage{graphicx}
\usepackage{mathtools}

\usepackage{booktabs}

\makeatletter
\newcommand*{\currentname}{\@currentlabelname}
\makeatother

\usepackage{fontspec}
% \setmainfont{UbuntuMono}
\newfontfamily\Bera{Bitstream Vera Sans Mono}[Scale=0.85]

\definecolor{mDarkTeal}{HTML}{23373b}
\definecolor{mLightBrown}{HTML}{EB811B}
\definecolor{mDarkBlue}{HTML}{0F3470}
\definecolor{mDarkGrey}{HTML}{999999}
\definecolor{mLighterGrey}{HTML}{CCCCCC}
\definecolor{mLightGrey}{rgb}{0.95,0.95,0.95}
\definecolor{mLightBlue}{HTML}{212751}

\newcommand{\lb}[1]{{\color{mLightBrown}#1}}

\newcommand{\bblock}[3]{
        \begin{block}{#1}
        #2
    \end{block}
}

\setbeamercolor{block title}{fg=mDarkTeal}
\setbeamercolor{frametitle}{bg=mDarkBlue, fg=mLightGrey}

\title{\LaTeX\ Introduction \hspace*{\fill}\includegraphics[width=4cm]{../../uzhpres/tex/uzh_logo_e_pos.pdf}}
\author{Philipp Gloor\inst{1}}
\institute[University of Zurich] % (optional)
{
  \inst{1}%
  philipp.gloor@gmail.com
  \\[\medskipamount]
      
}
\date{Septemer 2017} % (optional)



\subject{\LaTeX}

\begin{document}
    
    \begin{frame}
    \titlepage
    \end{frame}
\begin{frame}
\frametitle{About me}

\begin{block}{Education}
    \begin{itemize}
        \item 2012 -- Bachelor of Science UZH in Physics
        \item 2016 -- Master of Science UZH in Computational Science
    \end{itemize}
\end{block}

\begin{block}{Work}
    \begin{itemize}
        \item 2014 -- 2016 Software engineer CERN (remote)
        \item 2016 -- now PDF Tools AG
    \end{itemize}
\end{block}

\begin{block}{Email}
\begin{itemize}
    \item philipp.gloor@gmail.com
\end{itemize}
    
\end{block}
 
% In this slide, some important text will be
% \alert{highlighted} beause it's important.
% Please, don't abuse it.
 
% \begin{block}{Remark}
% Sample text
% \end{block}
 
% \begin{alertblock}{Important theorem}
% Sample text in red box
% \end{alertblock}
 
% \begin{examples}
% Sample text in green box. "Examples" is fixed as block title.
% \end{examples}
\end{frame}
\begin{frame}[t]\frametitle{Round of introduction}
    \begin{itemize}
        \item Name
        \item Occupation
        \item Any experience with \LaTeX\ so far?
        \item Expectations
    \end{itemize}
\end{frame}

    \begin{frame}{Basics of \LaTeX}
        \begin{itemize}
            \item What is needed to work with \LaTeX (distribution, editor)
            \item Creating a minimal document
            \item Document classes
            \item Loading and using packages
        \end{itemize}
    \end{frame}
    

    \begin{frame}{Technalities about \LaTeX}
    \LaTeX\ works differently than programs like e.g. Word which is a WYSIWYG editor (what you see is what you get). You will see examples of such occurences like:

    \begin{itemize}
         \item Change font face
         \item Change paragraph behaviour
         \item Change line spread
     \end{itemize}

     You will also understand how \LaTeX\ commands work and how to interpret the most common error messages.
    \end{frame}

    \begin{frame}{Structuring a document}
    What are the options to structure a \LaTeX\ document? Apart from simple sectioning commands there are many options that help a reader to navigate through the document - for example:
    \begin{itemize}
         \item Chapter/Section/Subsection etc
         \item Table of contents
         \item In-document references
         \item Link annotations for navigating the document
     \end{itemize}
    \end{frame}

    \begin{frame}{Layout of a document}
        \begin{itemize}
            \item Generally: Changing layout should be done sparsely
        \end{itemize}
    Changing the layout should be done very carefully in \LaTeX\ because one of the main reasons of why \LaTeX\ is so successful is because it does most of the layout itself and it does it very well. However there are cases where the writer wants to change a few things like the font face, the page dimensions or other small things.
    \end{frame}

    \begin{frame}[t]{Typesetting math formulas}
    \vspace{1cm}
    \[ \rho\left( \frac{\partial \mathbf{v}}{\partial t} + \mathbf{v}\cdot\nabla\mathbf{v}\right) = -\nabla p + \nabla\cdot\mathsf{T}+\mathbf{f} \]

    \vspace*{\fill}
    One of the core features of \LaTeX\ is the setting of mathematical formulas. Once the quirks are understood this becomes very easy.
    \vspace*{\fill}
    \end{frame}
    

    \begin{frame}{Other content}
    \begin{itemize}
        \item Tables
        \item Figures
        \item Images
    \end{itemize}
    \end{frame}

    
    \begin{frame}{Bibliography}
    Creation of a bibliography that will be placed into the document. This includes the whole workflow of creating a database with cited sources to the compilation of the \LaTeX\ document which then contains references and a bibliography.

    The course covers many different small topics with small examples. Following is an outline of areas in which these examples can be categorized in.
    \end{frame}





    
\end{document}