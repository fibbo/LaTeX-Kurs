\documentclass[10pt, a4paper,landscape]{article}
\usepackage[T1]{fontenc}
\usepackage[utf8]{inputenc}
\usepackage[a4paper, left=1cm, right=1.6cm, top=1cm, bottom=1.6cm]{geometry}
\usepackage[ngerman]{babel}
\usepackage{blindtext}
\usepackage[singlelinecheck = off]{caption} %damit caption an der tabelle ausgerichtet ist
\usepackage{threeparttable, booktabs} %booktabs fuer \bottomrule \toprule
\usepackage{multirow}
\usepackage[table]{xcolor}
\usepackage{lipsum}


\setlength{\abovecaptionskip}{3pt plus 3pt minus 2pt} 


%\setlength{\tabcolsep}{8pt}
\renewcommand{\arraystretch}{1.1} %Abstand zwischen zwei Zeilen

\setlength{\parindent}{0pt}

\begin{document}


\section{Ein erstes Beispiel}
Diese Tabelle steht im Text.
\begin{tabular}[c]{c|c}
\hline
1 & 2 \\
\hline 
3 & 4\\
\hline
\end{tabular}. Die Tabelle ist so kein freistehendes Objekt - das hat vor und Nachteile.

\section{Multirow und -column}

\setlength{\tabcolsep}{5pt} %Abstand zwischen zwei Spalten

\subsection{Multicolumn}

\begin{tabular}{lccc}
\hline
\textbf{\LaTeX-Befehl} & \multicolumn{3}{c}{\textbf{Basisschriftgrösse}}\\
\cline{2-4} & \textbf{10pt} & \textbf{11pt} & \textbf{12pt}\\
\hline
\end{tabular}

\subsection{Multirow}

\begin{tabular}{lll}
\hline
\multirow{3}{*}{Multirow} & 1 & 2 \\
 & a & b \\
 & c & d \\
\hline
\end{tabular}

\subsection{Multirow and -column combined}

\begin{tabular}{lccc}
\hline
\multirow{2}{5cm}{\centering\textbf{\LaTeX-Befehl}} & \multicolumn{3}{c}{\textbf{Basisschriftgr"oesse}} \\
& \textbf{10pt} & \textbf{11pt} & \textbf{12pt} \\
\hline
\end{tabular}

\subsection{Cline}

\begin{tabular}{|c|c|c|c|}
\hline
1 & 2 & 3 & 4\\
\cline{2-4} d & 10 & 11 & 12\\
\hline
\end{tabular}

{
\rowcolors{2}{white}{red!10}
\begin{tabular}{ l | p{20cm}  r }
  \textbf{a} & \textbf{b} & \textbf{c} \\
  \hline \hline
  16:00 -- 17:00 & \lipsum[1] & blubb \\
  17:00 -- 18:00 & Mehr Text & Endlich Ende \\
  18:00 -- 19:00 & \multicolumn{2}{r}{Endlich Abendessenszeit} \\
\end{tabular}\\[1em]
}
\section{Gleitumgebungen}

\begin{table}[h]
\caption{\blindtext}
\centering
\begin{tabular}{ll}
\hline
Name: & Baggins \\
First Name: & Bilbo\\
Address: & Bagshot Row 56, 10982 Hobbiton\\
\hline
\end{tabular}
\end{table}

\begin{table}[h]
\centering\renewcommand\arraystretch{1.1}
\begin{threeparttable}
\caption{\blindtext}
\begin{tabular}{ll}
\toprule
Name: & Baggins \\
First Name: & Bilbo \\
Address: & Bagshot Row 56, 10982 Hobbiton \\
\bottomrule
\end{tabular}
\end{threeparttable}
\end{table}


\end{document}
