\documentclass[10pt, a4paper]{article}
\usepackage[T1]{fontenc}
\usepackage[utf8]{inputenc}
\usepackage[ngerman]{babel}
\usepackage{blindtext}

\setlength{\parskip}{1.5ex plus 0.5ex minus 0.2ex}

\begin{document}
\linespread{2}\selectfont
\blindtext

\linespread{1}\selectfont
Man muss bei LaTeX wissen, dass eine neue Zeile und ein neuer Absatz nicht das gleiche sind. Um zum Beispiel eine neue Zeile zu beginnen braucht es zwei Backslashes \texttt{\textbackslash\textbackslash}.\\ Der Text der dann folgt wird auf eine neue Zeile geschrieben.
Einfach eine neue Zeile im Editor hat keinen Einfluss. Alternativ zu dem doppelten Backslash geht auch der Befehl \verb+\newline+ \newline Dieser beginnt ebenfalls eine neue Zeile.

Wenn man einen neuen Absatz will, braucht es eine Leerzeile, das heisst im Editor gibt es dann eine Zeile auf der nichts steht. Je nach Einstellungen werden neue Abs"atze durch einen Einzug der ersten Zeile oder manchmal einfach durch einen vergr"osserten Zeilenabstand zwischen den Abs"atzen dargestellt.

Auch hier gibt es wieder einen alternativen Befehl und zwar \verb+\par
+ \par Dieser Befehl beginnt ebenfalls ein neuer Absatz, ohne, dass man im \LaTeX-Dokument eine Leerzeile braucht.

\end{document}
